\chapter{Solution Design}
\label{sec:solution}

    •	What is the physical structure of cotton candy?
    What aspects define “quality” of cotton candy?     •	What are the key factors that affect quality?
    •	How does it change with production parameters?
    What we already learned doing cc is -> The notes from the notion

\section{Design Goals and Approach}
% High-level aims (why a prescriptive DT, bottom-up design, reproducibility).

\section{System Architecture}
% Conceptual block diagram: sensors, machine, twin, data collection.

\section{Sensor Design and Placement}
\subsection{Temperature Sensors}
% Infrared + ambient.


\paragraph{Infrared Object Temperature Sensor}
To monitor the thermal behavior of the cotton candy machine's rotating head, we use a non-contact infrared temperature sensor (MLX90614). This sensor is well suited for the application due to its high accuracy (±0.5°C in the critical 0–50°C range) and wide measurement span, covering object temperatures from --70°C up to 380°C. These properties make it ideal for tracking the heating phase where the sugar melts and begins to form cotton candy.

Although only one sensor is used during regular operation, we conducted a dedicated experiment using a second sensor to evaluate placement and correlation. One sensor was positioned inside the machine, 3 cm above the rotating head, while the other was mounted below the machine base, pointing sideways toward the head at the same distance. Over 2000 seconds, both sensors recorded temperature data to assess the relationship between internal and external readings.

\begin{figure}[h]
    \centering
    \includegraphics[width=0.85\linewidth]{figures/IrO1 and IrO2 Over Time.png}
    \caption{Comparison of internal (IrO1) and external (IrO2) infrared temperature sensors over 2000 seconds.}
    \label{fig:ir-comparison}
\end{figure}

A linear regression analysis showed a strong correlation between the two sensors, with a coefficient of determination $R^2 = 0.996$, slope $a = 0.383$, and intercept $b = 12.15$. The resulting model for estimating the external temperature $T_{\text{outside}}$ based on the internal reading $T_{\text{inside}}$ is:

\[
T_{\text{outside}} = 0.383 \cdot T_{\text{inside}} + 12.15
\]

This strong correlation implies that the internal head temperature can be reliably estimated from the external measurement. Consequently, the system design employs only a single infrared temperature sensor mounted outside the head during operation. This placement avoids obstructing the cotton candy formation process while maintaining accurate and reliable thermal monitoring. The sensor’s communication protocol and technical specifications are described in the official documentation~\cite{waveshareIRsensor}.

\subsection{Humidity Sensors}
% Inside vs outside, offset explanation.

\subsection{Humidity Sensor Offset}
During idle phases, the inside humidity sensor consistently showed 2--5\% higher values than the outside reference. This effect is best explained by placement and airflow, since it disappeared once the machine was running. Replacing the sensors produced the same pattern, confirming it is not sensor wear but an environmental effect. For the digital twin, this means that differences observed during operation can be interpreted as true process dynamics, such as moisture accumulation, rather than calibration bias.

\subsection{Energy Monitoring}
% Plug sensor and rationale.

\section{Cotton Candy Product Characteristics}
\subsection{What we learned empirically doing CC}
    - write times, forms, radius why that radius etc etc etc
    => We can create a formula of how we think it behaves, that we can compare afterwards in the Data Recollection Evaluation.
\subsection{Physical Structure}
% Amorphous vs crystalline, thermal study.

\subsection{Quality Indicators}
% Volume density, compression test, etc.

\subsection{Excluded Factors}
% Taste, visual appearance, long-term hygroscopic behavior.

\section{Process Parameters and Environmental Factors}
\subsection{Humidity}
% Effects during spinning and after; airflow.

\subsection{Temperature}
% Limited control, cooking time as proxy.

\subsection{Spinning Speed}
% Fixed.

\subsection{Sugar Amount and Formulation}
% Fixed, supermarket sugar.

\section{Prescriptive Digital Twin Design}
% How environment + process map to quality.
% Forecasting with decision trees.
% High-level evaluation plan.


\section{    •	What is the physical structure of cotton candy?}
Cotton candy is primarily composed of spun sucrose, which can exist in two main forms:WHat we saw in the paper\dots

\section{What aspects?}

YES: Volume density -> How much sugar is in a given volume -> Weighing sample and measuring volume (water displacement) => Doing this wiht a trichter that has the exact volume written. How does it change wiht more or less humidity?

NO: Visual appearance -> Fiber structure, color, consistency -> Visual inspection, photography + image analysis (even with your phone and Python/OpenCV) => we are not gonna do this because we learned that it is not really possible to distinguish the fibers with the naked eye. Its a full master thesis on its own

NO: Texture \& mouthfeel -> Stickiness, softness, “melt-in-mouth” -> Manual touch test, break force => Stickiness is interesting to measure but probably difficult more on it later 
NO: Hygroscopic behavior -> Stickiness as it absorbs moisture -> Weighing sample over time at room humidity => This is very interesting, takes time to measure but hey -> NO BC WE WILL CREATE THEM IN DIFF ENVIRONMENTS AND CANNOT CREATE A CONTROL CAPSULE

YES: Crispness vs softness -> Related to crystallinity -> Compression test (kitchen scale or small force sensor) => It would mean measureing how much compression force is needed to break the fibers, very difficult, we build a model that we could test between CC and after measuring volume we weighted it, we tested this and that and concluded\dots

 B. Compression Test
	•	Use a small kitchen scale or force sensor.
	•	Press gently until collapse starts.
	•	Record maximum weight/force applied.
	•	Amorphous cotton candy tends to be softer; more crystalline samples resist compression.
% ➡ Proxy for structure & texture.

NO: Structural stability -> How long it holds shape over time -> Timed visual check at room conditions => Takes long to test, 

NO: Taste (subjective) -> Flavor preception is too subjective so we will not do this, but it is important to note that it is a factor in quality perception.

\section{    •	How does it change with production parameters?}
How can we change the environment and control so that production parameters are changed 
    •	Humidity: Higher humidity leads to more stickiness and faster recrystallization. -> How to simulate Humidity?
    The gas environment during spinning directly changes how much of the sucrose becomes crystalline vs amorphous.
	•	More oxygen \& moisture → more crystallization.
	•	Less oxygen \& low moisture (like nitrogen or dry air) → more amorphous content.
    -> We cannot change the environment, but we can change the humidity of the process with:
        - Airflow / fans -> Stronger air movement near spinner -> Helps dry fibers during flight/creation, reduces moisture pickup.
    -> YESSSS LETS DO THIS

    % IMPORTANt
    Affect on more humidity durign spinning -> Fibers may break sooner, become shorter, thicker. fibers stick together more.
    and after spinning -> Fibers collapse and shrink, Loss of volume (shrinkage), stickiness increases. what about compression?
    Immediately after spinning (fresh)
- Fibers in humid air are thicker, stickier → denser structure → higher compression resistance initially (less fluffy, more “compact”). - Less air trapped between fibers → more force needed to compress.
Shortly after spinning (as moisture is absorbed)
- As fibers absorb moisture, they soften → compression force quickly drops. - Structure collapses under small loads.
% IMPORTANt

Compression is kind of complicated to test since: 
Actually, at high humidity:
	•	At first → more compact = higher compression resistance.
	•	But as time passes → absorbs moisture → weaker structure = lower compression resistance.

In contrast, at low humidity:
	•	The fibers stay dry, fluffy, and light.
	•	Lower compression resistance but much better structural stability over time.

    Humidity Level
        Result that we think we could achieve:
        Low RH \(<30\%\)
        Light, fluffy, large volume, fine fibers.
        High RH \(>60\%\)
        Denser, smaller volume, coarser fibers, faster shrinkage.


    •	Temperature: Higher temperatures can lead to more amorphous structure, but too high can cause burning. We have no control over this, as we are gonna see in the Solution Design, we are using a machine that always stays at the same ratio of temperature when at work. 
    -> What we can control and change is the Cooking time, so the temperature that the head had when inserting the suagr and the time that we let the cotton candy get created, and let the arm roll. -> What this hopefully gives us is the change in structure that we can test with the compression test and a bigger volume.

    •	Spinning speed: Faster spinning may lead to finer fibers, affecting texture. Spinning speed (Higher RPM) Creates finer fibers, helps counteract thickening effect of humidity. -> We cannot control this. The given machine always spins in the same speed.

    -  One big variable that I had at the start was the Sugar amount. Thinking naively, I thought that more sugar would lead to more volume, but this is not the case. The amount of sugar in the process is always the same, and we are not changing it. We are always using the same amount of sugar for each production, which is 10 grams. -> Adding more won’t help volume, might increase stickiness. And we are not measuring this change in stickiness as we saw before.

    - Sugar formulation -> Use anti-hygroscopic additives (e.g., small \% of maltodextrin or stabilizers) -> Slows moisture absorption. Often used in industrial production. We are not chanign the sugar formulation. We are always using the same sugar from the supermarket for making it easier to reproduce the results.



\section{Prescriptive Digital Twin Flow}

- We have an Environment that we cannot control, but we can measure it.
- We have a process that we can control, but what exactly? THe time of cook? (Yes) The amunt of sugar? (Yes, but doesnt impact), The heating up before spinning? (Yes, but only energy savings impact) The \dots
- We have a product that we can measure and evaluate the quality, but how? (First measure the Volume and then the compression stress, etc) 

With 200 points of this data we can create a Model that can forecast the quality mark of the product looking at the ENV and the Process Rules. With Desicion Trees since this and that\dots

With the Forecasting we can change the Process rules to imporve the quality of the product. How? With Decision Trees? Trasvering back? How should it be done? I dont know yet.


\begin{figure}[h]
    \centering
    \caption{My Figure Caption}
    \includegraphics[width=0.7\textwidth]{tum-resources/images/Universitaet_Flaggen.jpg}
    \floatfoot{A note describing the figure}
    \label{fig:firstFigure}
\end{figure}




%Maybe Trash
\section{Thermal Study on Cotton Candy}
Cotton candy consists of spun sucrose that cools rapidly, forming a mostly amorphous structure — but:
	•	Over time, this amorphous state can convert into crystalline form.
	•	The ratio between crystalline and amorphous sucrose affects:
	•	Texture
	•	Stability
	•	Taste
	•	Shelf-life

% The paper investigates how the physical structure of cotton candy changes depending on how it is produced and stored, focusing especially on how much is amorphous vs crystalline sucrose.
% 	•	The author studies this using Differential Scanning Calorimetry (DSC), a thermal analysis technique to measure transitions like glass transitions, crystallization, and melting.
% 	•	The novelty: the study compares cotton candy spun in air vs in nitrogen atmosphere to see how different gas environments affect its physical and thermal properties.

% In essence, it’s a study of the physics of sugar in cotton candy — very relevant to your Cotton Candy Automata project, as this kind of data gives insight into the material side of your digital twin.

This paper provides very solid experimental data on how production parameters influence the physical structure (crystalline vs amorphous) of cotton candy.

Crystalline is \dots
Amorphous is \dots

% Because the amorphous vs crystalline ratio affects:
% 	•	Shelf life (amorphous recrystallizes over time)
% 	•	Stability (stickiness, collapse, hardening)
% 	•	Thermal behavior (different melting/glass transition points)
% 	•	Quality (texture, mouthfeel)

Knowing how production parameters (like gas environment) affect structure can inform optimal production recipes. We are not gonna compare CCA vs CCN, since we are always using Normal Air but
We learn about the importance of humidity, and want to use it for our Data Recollection since this is important For the Prescriptive twin design.

This helps with us taking the decision how to measure the quality of CC when doing data recollection and giving a note to the process.

What makes it difficult is that the changes are fine and little, and we dont really know if we are gonna be able to distinguish them, but we did research and will introducte this in the Solution Design.


