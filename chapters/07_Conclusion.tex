\chapter{Conclusion}
\label{sec:conclusion}

1-2 pages

% And finally Conclusion, where we summarize the key contributions of our work, and its significance in the broader context of digital twin research and applications.

% Restate the Motivation & Approach
% 	•	Start by briefly recalling why prescriptive digital twins matter (efficiency, sustainability, quality improvements).
% 	•	Emphasize that your work focused on building a bottom-up prescriptive digital twin for the Cotton Candy Automata.

% ⸻

% 2. Summarize Key Findings

% Answer each research question in a concise way:
% 	•	RQ1: The digital twin improved production quality, reduced energy use, and optimized time compared to the baseline “unintelligent” automata.
% 	•	RQ2: Analysis of process data revealed strong correlations (e.g., between heating time and candy texture, or sugar amount and production consistency), confirming that empirical parameter identification was effective.
% 	•	RQ3: By mapping the methodology to popcorn production, you showed that the bottom-up approach is transferable to other thermo-transformative processes, reinforcing its generalizability.
% 	•	RQ4: The prescriptive twin was able to recommend process adjustments, though environmental factors such as ambient humidity could not yet be controlled. These represent promising directions for future improvement.

% ⸻

% 3. Contributions
% 	•	Practical: A working prototype of a prescriptive digital twin for a robotic system, showing measurable efficiency and quality gains.
% 	•	Methodological: Demonstrated how a bottom-up, sensor-driven approach can be used to construct prescriptive digital twins, reducing the need for large historical datasets.
% 	•	Conceptual: Provided a transferable abstraction (inputs, controls, states, outputs, constraints) that aligns with the Research Manifesto on Digital Twins of Business Processes (Fornari et al., 2025) and can be applied in other domains.

% ⸻

% 4. Limitations
% 	•	Depended on one specific robotic system in a lab setting.
% 	•	Environmental factors (humidity, temperature) could only be observed, not manipulated.
% 	•	Scaling to more complex industrial processes would require integration with additional sensors and actuators.

% ⸻

% 5. Future Work
% 	•	Implementing environmental control (humidity/temperature) to test full prescriptive capabilities.
% 	•	Extending the methodology to other processes (e.g., roasting, baking, drying).
% 	•	Incorporating advanced AI techniques (e.g., reinforcement learning) for adaptive optimization, as recommended in the digital twin literature .
% 	•	Integrating human-in-the-loop features to improve explainability and usability.

% ⸻

% 6. Closing Statement

% Conclude with something like:

% Overall, this thesis demonstrates the feasibility and value of bottom-up prescriptive digital twins for small-scale production systems. By combining empirical process data with prescriptive decision-making, the approach contributes to more efficient, sustainable, and adaptive operations. Beyond the cotton candy use case, the findings highlight pathways for generalizing digital twin design to broader industrial contexts, offering a foundation for future research and real-world applications.