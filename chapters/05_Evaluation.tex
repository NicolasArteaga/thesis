\chapter{Evaluation}
\label{sec:evaluation}

\section{Product Output Quality}

Test for github of changing things here

% Stichpunkte:
% - You refer to the volume measurements to analyze quality.
% - You compare volume/fluffiness before and after optimization.
% - Do not repeat the formula or measurement steps here.

The quality of the cotton candy produced in each run was assessed using the previously introduced weight and volume-based metrics. In particular, changes in product volume and the derived Fluffiness Index were analyzed across multiple runs to evaluate whether the digital twin contributed to a measurable improvement in output quality.

\section{Evaluation Methodology and Test Batches}

To validate the effectiveness of the digital twin system, we conducted a comprehensive evaluation using dedicated test batches separate from the training data. This evaluation phase, commonly referred to as \textit{test set evaluation} in machine learning, ensures unbiased assessment of the digital twin's predictive capabilities and optimization performance.

The evaluation methodology consisted of three distinct phases:

\textbf{Phase 1: Predictive Accuracy Assessment}
We executed a series of production runs while monitoring environmental conditions and process parameters in real-time. For each run, the digital twin system predicted the expected quality score before production began. These predictions were then compared against the actual measured quality metrics (weight, volume, and derived Fluffiness Index) to assess the accuracy of the digital twin's predictive model. This comparison allowed us to quantify the prediction error and evaluate the reliability of the quality forecasting capabilities.

\textbf{Phase 2: Comparative Analysis with Baseline System}
In parallel, we implemented and tested the simple control function originally developed for the CPEE (Cloud Process Execution Engine) system. This baseline approach relies on basic parameter thresholds without machine learning optimization. For identical environmental conditions and input parameters, we compared the performance of both systems in terms of quality prediction accuracy and optimization recommendations. This comparative analysis demonstrates the added value of the digital twin approach over conventional rule-based control systems.

\textbf{Phase 3: Environment Replication and Validation}
The final evaluation phase involved replicating specific environmental conditions that occurred during previous production runs. Using the recorded sensor data from earlier batches, we recreated similar temperature, humidity, and process parameter combinations. The digital twin system then provided optimization recommendations for these replicated scenarios, and we executed production runs following these recommendations. The resulting quality metrics were compared against both the digital twin's predictions and the actual historical results from the original conditions, validating the system's ability to reproduce and improve upon past performance.

Each evaluation batch consisted of multiple production runs under controlled conditions, with comprehensive data logging to ensure reproducible results. The statistical analysis of these evaluation batches provides quantitative evidence of the digital twin's effectiveness in cotton candy quality optimization and process control reliability.

\section{Product Output Consistency}
Like we saw in the research, the less the max pressure of the sugar the better quality the cotton candy.

the smaller the cc, the less quality (when count is )

After count >= 10 its a bad score (too small)